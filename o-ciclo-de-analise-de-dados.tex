% Options for packages loaded elsewhere
\PassOptionsToPackage{unicode}{hyperref}
\PassOptionsToPackage{hyphens}{url}
%
\documentclass[
  letterpaper,
  paper=6in:9in,
  pagesize=pdftex,
  headinclude=on,
  footinclude=on,
  12pt]{scrbook}

\usepackage{amsmath,amssymb}
\usepackage{iftex}
\ifPDFTeX
  \usepackage[T1]{fontenc}
  \usepackage[utf8]{inputenc}
  \usepackage{textcomp} % provide euro and other symbols
\else % if luatex or xetex
  \usepackage{unicode-math}
  \defaultfontfeatures{Scale=MatchLowercase}
  \defaultfontfeatures[\rmfamily]{Ligatures=TeX,Scale=1}
\fi
\usepackage{lmodern}
\ifPDFTeX\else  
    % xetex/luatex font selection
\fi
% Use upquote if available, for straight quotes in verbatim environments
\IfFileExists{upquote.sty}{\usepackage{upquote}}{}
\IfFileExists{microtype.sty}{% use microtype if available
  \usepackage[]{microtype}
  \UseMicrotypeSet[protrusion]{basicmath} % disable protrusion for tt fonts
}{}
\makeatletter
\@ifundefined{KOMAClassName}{% if non-KOMA class
  \IfFileExists{parskip.sty}{%
    \usepackage{parskip}
  }{% else
    \setlength{\parindent}{0pt}
    \setlength{\parskip}{6pt plus 2pt minus 1pt}}
}{% if KOMA class
  \KOMAoptions{parskip=half}}
\makeatother
\usepackage{xcolor}
\setlength{\emergencystretch}{3em} % prevent overfull lines
\setcounter{secnumdepth}{5}
% Make \paragraph and \subparagraph free-standing
\ifx\paragraph\undefined\else
  \let\oldparagraph\paragraph
  \renewcommand{\paragraph}[1]{\oldparagraph{#1}\mbox{}}
\fi
\ifx\subparagraph\undefined\else
  \let\oldsubparagraph\subparagraph
  \renewcommand{\subparagraph}[1]{\oldsubparagraph{#1}\mbox{}}
\fi


\providecommand{\tightlist}{%
  \setlength{\itemsep}{0pt}\setlength{\parskip}{0pt}}\usepackage{longtable,booktabs,array}
\usepackage{calc} % for calculating minipage widths
% Correct order of tables after \paragraph or \subparagraph
\usepackage{etoolbox}
\makeatletter
\patchcmd\longtable{\par}{\if@noskipsec\mbox{}\fi\par}{}{}
\makeatother
% Allow footnotes in longtable head/foot
\IfFileExists{footnotehyper.sty}{\usepackage{footnotehyper}}{\usepackage{footnote}}
\makesavenoteenv{longtable}
\usepackage{graphicx}
\makeatletter
\def\maxwidth{\ifdim\Gin@nat@width>\linewidth\linewidth\else\Gin@nat@width\fi}
\def\maxheight{\ifdim\Gin@nat@height>\textheight\textheight\else\Gin@nat@height\fi}
\makeatother
% Scale images if necessary, so that they will not overflow the page
% margins by default, and it is still possible to overwrite the defaults
% using explicit options in \includegraphics[width, height, ...]{}
\setkeys{Gin}{width=\maxwidth,height=\maxheight,keepaspectratio}
% Set default figure placement to htbp
\makeatletter
\def\fps@figure{htbp}
\makeatother

\usepackage{fvextra}
\DefineVerbatimEnvironment{Highlighting}{Verbatim}{breaklines,commandchars=\\\{\}}
\areaset[0.50in]{4.5in}{8in}
\makeatletter
\@ifpackageloaded{bookmark}{}{\usepackage{bookmark}}
\makeatother
\makeatletter
\@ifpackageloaded{caption}{}{\usepackage{caption}}
\AtBeginDocument{%
\ifdefined\contentsname
  \renewcommand*\contentsname{Índice}
\else
  \newcommand\contentsname{Índice}
\fi
\ifdefined\listfigurename
  \renewcommand*\listfigurename{Lista de Figuras}
\else
  \newcommand\listfigurename{Lista de Figuras}
\fi
\ifdefined\listtablename
  \renewcommand*\listtablename{Lista de Tabelas}
\else
  \newcommand\listtablename{Lista de Tabelas}
\fi
\ifdefined\figurename
  \renewcommand*\figurename{Figura}
\else
  \newcommand\figurename{Figura}
\fi
\ifdefined\tablename
  \renewcommand*\tablename{Tabela}
\else
  \newcommand\tablename{Tabela}
\fi
}
\@ifpackageloaded{float}{}{\usepackage{float}}
\floatstyle{ruled}
\@ifundefined{c@chapter}{\newfloat{codelisting}{h}{lop}}{\newfloat{codelisting}{h}{lop}[chapter]}
\floatname{codelisting}{Listagem}
\newcommand*\listoflistings{\listof{codelisting}{Lista de Listagens}}
\makeatother
\makeatletter
\@ifpackageloaded{caption}{}{\usepackage{caption}}
\@ifpackageloaded{subcaption}{}{\usepackage{subcaption}}
\makeatother
\makeatletter
\makeatother
\ifLuaTeX
\usepackage[bidi=basic]{babel}
\else
\usepackage[bidi=default]{babel}
\fi
\babelprovide[main,import]{portuguese}
% get rid of language-specific shorthands (see #6817):
\let\LanguageShortHands\languageshorthands
\def\languageshorthands#1{}
\ifLuaTeX
  \usepackage{selnolig}  % disable illegal ligatures
\fi
\IfFileExists{bookmark.sty}{\usepackage{bookmark}}{\usepackage{hyperref}}
\IfFileExists{xurl.sty}{\usepackage{xurl}}{} % add URL line breaks if available
\urlstyle{same} % disable monospaced font for URLs
\hypersetup{
  pdftitle={O ciclo de análise de dados},
  pdfauthor={Análise Macro},
  pdflang={pt},
  hidelinks,
  pdfcreator={LaTeX via pandoc}}

\usepackage{titling}
\let\cleardoublepage\clearpage

\title{\textbf{O ciclo de análise de dados}}
\subtitle{Um roteiro completo para resolver problemas do dia a dia}
\author{Um roteiro completo para resolver \\ problemas do dia a dia}
\date{}

\pretitle{%
	\begin{center}
		\LARGE
		\includegraphics[width=4cm,height=6cm]{imagens/logo_am.png}\\[\bigskipamount]
	}
\posttitle{\end{center}}
\begin{document}
\frontmatter
\maketitle
\RecustomVerbatimEnvironment{verbatim}{Verbatim}{
   showspaces = false,
   showtabs = false,
   breaksymbolleft={},
   breaklines
   % Note: setting commandchars=\\\{\} here will cause an error 
}

\renewcommand*\contentsname{Índice}
{
\setcounter{tocdepth}{2}
\tableofcontents
}
\mainmatter
\bookmarksetup{startatroot}

\chapter*{Bem vindo(a)!}\label{bem-vindoa}
\addcontentsline{toc}{chapter}{Bem vindo(a)!}

\markboth{Bem vindo(a)!}{Bem vindo(a)!}

Seja bem vindo(a) ao ebook \textbf{O ciclo de análise de dados}
produzido pela \href{https://analisemacro.com.br/}{\textbf{Análise
Macro}}.

A demanda por análise de dados é cada vez mais crescente nas empresas,
no governo e na academia. Existem muitos problemas que podem ser
resolvidos usando a união correta de técnicas, habilidades, dados e
ferramentas no dia a dia dos profissionais da área. Nesse ebook você vai
aprender sobre um roteiro separado em etapas que agilizam o processo de
analisar dados, facilitando a entrega de soluções baseadas em dados.
Aprenda a metodologia de trabalho, as aplicações e exemplos de uso de
análise de dados no mundo contemporâneo.

Para se aprofundar e aprender mais confira a formação
\href{https://analisemacro.com.br/formacao/do-zero-a-analise-de-dados-com-python/}{\textbf{Do
Zero à Análise de Dados com Python}} da Análise Macro.

Você pode ler a primeira versão deste ebook online, em PDF ou no formato
epub.

\bookmarksetup{startatroot}

\chapter{Introdução}\label{introduuxe7uxe3o}

Resolver problemas é uma tarefa central para quem trabalha na área de
Dados, especialmente em análise de dados.~

O papel do analista é utilizar suas habilidades em estatística,
matemática, programação e outras, além da \emph{expertise} da área, para
resolver um problema utilizando dados.~

Mas qual problema o Fernando, um analista de dados, vai resolver?~

Definir o problema a ser resolvido e os objetivos da análise de dados é
o primeiro passo fundamental para desenvolver um trabalho bem sucedido.~

Sem saber o que é necessário resolver é difícil que qualquer solução
desenvolvida atinja e resolva o problema.~

É preciso muita sorte para produzir boas análises de dados sem se guiar
por um propósito claro.

Portanto, o primeiro passo é utilizar uma metodologia com processos e
etapas bem definidas para analisar dados.

Neste ebook vamos descrever:

\begin{enumerate}
\def\labelenumi{\arabic{enumi})}
\tightlist
\item
  que metodologia para analisar dados é essa;
\item
  quais são as etapas gerais;
\item
  e como elas funcionam para resolver problemas reais.~
\end{enumerate}

Sem uma metodologia de trabalho é seguro dizer que o analista de dados
está perdido numa selva de ferramentas, modelos e dados, lutando para
sobreviver e tentando qualquer coisa a todo momento e a qualquer custo.~

Ao seguir uma metodologia de trabalho, o analista estará guiado por uma
bússola, o que diminui as chances de se perder no caminho e garante
consistência de resultados no longo prazo.

\bookmarksetup{startatroot}

\chapter{O ciclo}\label{o-ciclo}

\begin{figure}[H]

\centering{

\includegraphics{imagens/ciclo_dados.png}

}

\end{figure}%

O que chamamos de ciclo de análise de dados é uma metodologia de
trabalho para otimizar e guiar o processo de analisar dados, desde a
definição do problema a ser resolvido até a implementação da solução
baseada em dados.~

É um ciclo porque, na prática, resolver problemas com dados não é como
caminhar em linha reta do ponto A até o ponto B.~

O dia a dia de análise de dados é cheio de idas e vindas, tentativas e
erros, pois há muitas pedras no caminho e algumas são difíceis de
ultrapassar.~

Algumas, dentre várias, dessas pedras no caminho de um analista de dados
são:

\begin{itemize}
\item
  Dados indisponíveis;
\item
  Dados incorretos;
\item
  Dados ausentes;
\item
  Objetivos e problemas indefinidos;
\item
  Entre outros.
\end{itemize}

Alguns destes obstáculos para analisar dados podem ser melhor
contornados se houver uma visão clara do caminho a ser percorrido.~

Dessa forma, o ciclo de análise de dados é como um mapa que o analista
pode utilizar para pegar um problema, analisar os dados e entregar uma
solução.

\begin{description}
\item[\textbf{Por exemplo:}]
\emph{Se você é um analista econômico, o ciclo de análise de dados
compreende todo o processo de extração, processamento, análise e
apresentação de dados de conjuntura econômica, com vistas a fornecer
informações atualizadas e relevantes para o seu público alvo.}
\end{description}

Para saber mais sobre análise econômica, o curso de
\href{https://analisemacro.com.br/curso/analise-de-conjuntura-usando-python/}{\textbf{Análise
de Conjuntura usando Python}} ensina todas estas etapas de maneira
sistemática.

Entender a fundo o ciclo de análise de dados é fundamental para
conseguir entregar soluções e informações a partir de dados.~

Portanto, um analista de dados deve ser capaz de mapear mentalmente,
dado um contexto, essas etapas para desenvolver uma solução a partir de
dados.

Vamos dar uma olhada nas etapas?

\section{Objetivo}\label{objetivo}

\begin{figure}[H]

\centering{

\includegraphics{imagens/objetivo_ex.png}

}

\end{figure}%

É a primeira etapa de um projeto de análise de dados, onde há um
contexto/situação na área de atuação do analista de onde surge um
problema a ser resolvido.~

É papel do analista de dados, com apoio de outros atores envolvidos,
identificar esse problema de forma clara para prosseguir com uma solução
analítica de dados com determinados objetivos.

\begin{description}
\item[\textbf{Por exemplo:}]
\emph{Você é analista de dados na Netflix e o setor que monitora o
engajamento do usuário (tempo de uso, nº de títulos assistidos, etc.) no
serviço de streaming percebe uma queda em várias métricas, o que pode
ser um prenúncio de cancelamento de assinaturas.}
\end{description}

Nesse caso o problema é a queda de engajamento e o objetivo poderia ser
aumentar o engajamento com vistas a evitar cancelamento de assinaturas.

Nessa etapa é fundamental a \emph{expertise} de negócio para definir o
problema e os objetivos do projeto de análise de dados, além de ser
importante habilidades de comunicação interpessoal para contato com
outras pessoas técnicas e não-técnicas.

\section{Dados}\label{dados}

\begin{figure}[H]

\centering{

\includegraphics{imagens/dados_ex.png}

}

\end{figure}%

É a segunda etapa de um projeto de análise de dados, onde o objetivo é,
a partir de um problema definido, identificar quais dados podem ser
úteis para o desenvolvimento de uma solução.~

Os dados podem ser disponibilizados internamente ou externamente,
portanto essa etapa também compreende os procedimentos de coleta dos
dados necessários.

\begin{description}
\item[\textbf{Por exemplo:}]
\emph{No caso de queda de engajamento de usuários da Netflix, o analista
de dados poderia coletar internamente dados históricos de tempo de uso,
horas assistidas, categorias e temas de títulos assistidos,
atores/diretores do título, dados socioeconômicos como região, idioma,
gênero e etc. sobre os usuários.}
\end{description}

Externamente o analista de dados poderia coletar dados dos
\emph{players} concorrentes do mercado, se houver suspeitas que o
engajamento está sendo direcionado para outros serviços de
\emph{streaming}.

Nessa etapa já é necessário habilidades técnicas de programação,
consultas a bancos de dados, \emph{APIs} e outras para que os dados
possam ser disponibilizados para análise.

Para aprender a coletar dados de múltiplas fontes e formatos, confira o
curso de
\href{https://analisemacro.com.br/curso/programacao-em-python-para-analise-de-dados/}{\textbf{Programação
em Python para Análise de Dados}}.

Ferramentas comuns utilizadas nessa etapa são as linguagens de
programação R e Python e a linguagem de consulta SQL.~

\section{Exploração}\label{explorauxe7uxe3o}

\begin{figure}[H]

\centering{

\includegraphics{imagens/exploracao_ex.png}

}

\end{figure}%

Nessa etapa da análise de dados o objetivo é compreender o que está
acontecendo ou aconteceu com os dados, identificar padrões, relações e
anomalias que possam servir de sinal para a escolha de uma solução do
problema.~

Os dados precisam estar organizados para que possam ser analisados,
portanto é necessário transformar os dados brutos coletados previamente
para construir uma Tabela Analítica Base (ABT, no inglês), que servirá
para realizar a análise exploratória dos dados, desenvolver modelos
preditivos ou construir produtos de dados como relatórios e dashboards.

\begin{description}
\item[Por exemplo:]
\emph{No caso da Netflix, o analista de dados poderia fazer as limpezas
e cruzamentos de tabelas de dados necessárias, analisar a distribuição
das variáveis, identificar a variável ``alvo'' (aquela que é utilizada
para modelos preditivos, por exemplo), detectar valores ausentes,
verificar valores extremos ou outliers, analisar correlações e
autocorrelações dos dados, identificar tendências e sazonalidades,
dentre outras análises que podem ser úteis.}
\end{description}

Nessa etapa são fundamentais conhecimentos e habilidades em estatística,
programação e visualização de dados.

O conhecimento de estatística básica pode ser fundamental para se
destacar nesta etapa. O curso de
\href{https://analisemacro.com.br/curso/estatistica-para-analise-de-dados-usando-python/}{\textbf{Estatística
para Análise de Dados usando Python}} ensina todos os fundamentos
necessários.

As principais ferramentas utilizadas para essas análises são linguagens
de programação como R e Python, pacotes de tratamento e exploração de
dados como \texttt{tidyverse} e \texttt{pandas} e pacotes de
visualização de dados como \texttt{ggplot2} e \texttt{matplotlib}.

\section{Modelagem}\label{modelagem}

\begin{figure}[H]

\centering{

\includegraphics{imagens/modelagem_ex.png}

}

\end{figure}%

Nessa etapa o objetivo é levantar e experimentar possíveis soluções
baseadas em dados para o problema identificado previamente, podendo
ser:~

\begin{enumerate}
\def\labelenumi{\arabic{enumi}.}
\item
  Simples consultas SQL para agregar e sumarizar dados e informações;
\item
  Análises estatísticas como testes de hipótese, análise de regressão e
  outras;
\item
  Modelos econométricos para explicar relações, produzir inferências ou
  previsões;
\item
  Modelos preditivos com técnicas de \emph{machine learning}.~
\end{enumerate}

A técnica escolhida depende diretamente da definição do problema e dos
dados escolhidos, além de ser preferível, a depender do contexto,
técnicas/soluções simples e rápidas.~

No mundo real o tempo custa dinheiro e implementar algoritmos complexos
e avançados em produção gera uma fatura no final do mês que precisa ser
paga.

\begin{description}
\item[\textbf{Por exemplo:}]
\emph{No caso da Netflix, o analista poderia focar, por exemplo, em uma
solução de redução de Churn, identificando o perfil de usuários que
cancelaram a assinatura e prevendo a probabilidade de ocorrer o
cancelamento (risco de evasão), o que possibilita a tomada de decisão
para minimizar essa evasão de usuários.}
\end{description}

Em outras palavras, poderiam ser empregados modelos supervisionados de
classificação, usando técnicas de \emph{machine learning}.

Aprenda mais sobre os principais modelos e técnicas de previsão no curso
\href{https://analisemacro.com.br/curso/modelagem-e-previsao-usando-python/}{\textbf{Modelagem
e Previsão usando Python}} da Análise Macro.

Nessa etapa é fundamental o conhecimento de uma ampla gama de técnicas
estatísticas, econométricas e de \emph{machine learning}, domínio de
algoritmos e pacotes computacionais para implementar essas técnicas com
linguagens de programação, como o R e o Python e, dependendo do
contexto, conhecimento de ferramentas para processamento de Big Data.

\section{Validação}\label{validauxe7uxe3o}

\begin{figure}[H]

\centering{

\includegraphics{imagens/validacao_ex.png}

}

\end{figure}%

Nessa etapa o objetivo é avaliar se a solução analítica baseada em dados
é capaz de resolver o problema, podendo ser analisadas as métricas de
acurácia de modelos, os resultados estatísticos e econométricos de
testes ou ainda o \emph{feedback} do usuário/\emph{stakeholder} em caso
de soluções simples, como entrega de informações e \emph{insights} em
relatórios/dashboards.

\begin{description}
\item[Por exemplo:]
\emph{No caso da Netflix, o analista poderia analisar a acurácia de
diferentes modelos usando amostras de treino/teste, validação cruzada,
além de verificar a importância das variáveis utilizadas.}
\end{description}

O analista também deve ser capaz de fazer escolhas e tomar decisões sem
que isso prejudique ou deturpe os resultados encontrados.

Aprenda mais sobre métricas e avaliação de modelos de previsão no curso
\href{https://analisemacro.com.br/curso/modelagem-e-previsao-usando-python/}{\textbf{Modelagem
e Previsão usando Python}}.

Nesta etapa é fundamental o conhecimento em amostragem de dados,
interpretação estatística e programação usando linguagens como R e
Python.

\section{Implantação}\label{implantauxe7uxe3o}

\begin{figure}[H]

\centering{

\includegraphics{imagens/implantacao_ex.png}

}

\end{figure}%

Na última etapa do ciclo de análise de dados o objetivo é comunicar os
resultados do trabalho para os \emph{stakeholders} e usuários,
permitindo a tomada de decisão baseada em dados.~

Isso pode se traduzir na implementação em ambiente de produção de um
modelo preditivo, um sistema de recomendação, uma dashboard ou relatório
automatizado, dentre outras possibilidades.

\begin{description}
\item[\textbf{Por exemplo:}]
\emph{No caso da Netflix, o analista poderia elaborar uma apresentação
para os tomadores de decisão da companhia, permitindo a elaboração de
estratégias para reter os usuários que possuem alta probabilidade de
Churn.}
\end{description}

O modelo de classificação poderia, adicionalmente, ser implementado em
produção para, por exemplo, automaticamente recomendar títulos ou
oferecer descontos para usuários com probabilidade de evasão.

Aprenda mais sobre a produção de relatórios e apresentações
automatizadas com o curso
\href{https://analisemacro.com.br/curso/producao-de-relatorios-automaticos-usando-python/}{\textbf{Produção
de Relatórios Automáticos usando Python}}. Para ir além, confira o curso
de
\href{https://analisemacro.com.br/curso/producao-de-dashboards-automaticos-usando-python/}{\textbf{Produção
de Dashboards Automáticos usando Python}} que implementa diversos
exemplos de dashboards de análise de dados.

Nesta etapa é fundamental habilidades não técnicas de comunicação
interpessoal, apresentação e argumentação, além de habilidades técnicas
de infraestrutura e serviços de \emph{Cloud} e \emph{deploy} de modelos.

\bookmarksetup{startatroot}

\chapter{Conclusão}\label{conclusuxe3o}

O ciclo de análise de dados é vasto e complexo, mas ao mesmo tempo é uma
metodologia poderosa para solucionar problemas usando dados.~

O profissional que atua ou deseja atuar na área de dados precisa de
diversas habilidades e conhecimentos técnicos e não técnicos, de uma
ponta até a outra do ciclo, para agregar valor em uma empresa.~

Neste ebook apresentamos uma visão geral sobre o processo de análise de
dados, exemplos de aplicações e uso e as habilidades e ferramentas
necessárias para trabalhar na área.

\section{\texorpdfstring{\textbf{Quer aprender
mais?}}{Quer aprender mais?}}\label{quer-aprender-mais}

\href{https://analisemacro.com.br/boletim-am/}{\textbf{Clique aqui para
fazer seu cadastro no Boletim AM e baixar códigos de análise de dados}},
além de receber novos exercícios com exemplos reais envolvendo as áreas
de:

\begin{itemize}
\item
  \emph{Data Science}
\item
  Econometria
\item
  \emph{Machine Learning}
\item
  Macroeconomia Aplicada
\item
  Finanças Quantitativas e
\item
  Políticas Públicas
\end{itemize}

Toda semana novos exercícios aplicados usando Python, diretamente no seu
e-mail!


\backmatter

\end{document}
